%\documentclass[11pt,reqno]{amsart}
\documentclass[11pt,reqno]{article}
\usepackage[margin=.8in, paperwidth=8.5in, paperheight=11in]{geometry}
%\usepackage{geometry}                % See geometry.pdf to learn the layout options. There are lots.
%\geometry{letterpaper}                   % ... or a4paper or a5paper or ... 
%\geometry{landscape}                % Activate for for rotated page geometry
%\usepackage[parfill]{parskip}    % Activate to begin paragraphs with an empty line rather than an indent7
\usepackage{graphicx}
\usepackage{pstricks}
\usepackage{amssymb}
\usepackage{epstopdf}
\usepackage{amsmath}
\usepackage{subfigure}
\usepackage{caption}
\pagestyle{plain}
%\renewcommand{\topfraction}{0.3}
%\renewcommand{\bottomfraction}{0.8}
%\renewcommand{\textfraction}{0.07}
\DeclareGraphicsRule{.tif}{png}{.png}{`convert #1 `dirname #1`/`basename #1 .tif`.png}

\title{Real Analysis $\mathbb{I}$: \\ Assignment 4}
\author{Andrew Rickert}
\date{Started: April 8, 2011 \\ \hspace{1pt} Ended: April ??  2011}                                           % Activate to display a given date or no date

\begin{document}
\maketitle


% Page 1
\begin{flushleft} 
\textbf{Class 18.100B} - Problem 1\\
\rule{500pt}{1pt}\\
\end{flushleft} 

We are intended to find an open cover of $E = \{(x_1,x_2) | \mathbb{R}^2 : x_1^2 + x_2^2 < 1\} \subset \mathbb{R}^2$. Let the open cover be $C = \cup_{n = 1} U_n$ where $U_n = \{ (x_1,x_2) | \mathbb{R}^2 : x_1^2 + x_2^2 < 1-\frac{1}{n} \}$. Clearly $U_n \subset C$ for all $n \in \mathbb{N}$ but suppose $(x_1,x_2) \in E$ and $ x_1^2 + x_2^2  = r < 1$. Let $\epsilon < 1 - r$, and there must be a $n'$ such that $\frac{1}{n'} < \epsilon$ otherwise $\frac{1}{n} \ge \epsilon \implies \frac{1}{\epsilon} > n$ for all $n \in \mathbb{N}$. Since $\frac{1}{n'} < \epsilon < 1 - r$ this implies that $r < 1 - \frac{1}{n'}$. This means $(x_1,x_2) \in U_{n'}$ so $C \subset \cup U_n$ which implies that $C = \cup U_n$. So, $\cup U_n$ is an open cover. \\
\indent If we take a finite subset of these then $U_{n_1} \cup U_{n_2} \cdots \cup U_{n_m}$ is not an open cover. Let $p = $min$(n_1,n_2,\cdots,n_m)$ then $U_p$ is the 'largest' subset in the finite subcollection. Since $p < 1$ we can find an $r < 1 - p$ that is, there is an $r$ such that $p < 1 - r$. If we let $(x_1,x_2)$ be such that $x_1 = 0$ and $x_2 = \sqrt{1-r}$ then we have $(x_1,x_2) \in E$ but $x_1^2 + x_2^2 = 1 - r > p$ so $(x_1,x_2) \notin U_{n_1} \cup U_{n_2} \cdots U_{n_m}$. This must be true for any finite subcollection of $U_n$ and there is no finite subcover with the collection $U_n$.

\vspace{15pt}
\begin{flushleft} 
\textbf{Class 18.100B} - Problem 2\\
\rule{500pt}{1pt}\\
\end{flushleft} 

First we must show that $d(\mathbf{x},\mathbf{y}) = \| \mathbf{x} \| + \| \mathbf{y} \|$ when $\mathbf{x} \neq \mathbf{y}$ and $d(\mathbf{x},\mathbf{x}) = 0$ is a metric. Since $\|x\| = \sqrt{\sum_i x_i^2}$ we have $\| x \| \ge 0$ since $x_i^2 \ge 0$. If we assume that $\mathbf{x} \neq \mathbf{y}$ then $\|\mathbf{x}\| \neq 0$ so  $\| \mathbf{x} \| + \|  \mathbf{y} \| > 0$. Also $d(\mathbf{x},\mathbf{y}) = 0$ by definition when $\mathbf{x} = \mathbf{y}$.\\
\indent The symmetry property is straightforward since $d(\mathbf{x},\mathbf{y})  =  \| \mathbf{x} \| + \| \mathbf{y} \| =  \| \mathbf{y} \| + \| \mathbf{x} \|  = d(\mathbf{y},\mathbf{x})$.\\
\indent The triangle inequality is also easy since by the remark in the first paragraph $\| \mathbf{z} \| \ge 0$ we have $\| \mathbf{z} \| + \| \mathbf{z} \| \ge 0 \implies \| \mathbf{x} \|  + \| \mathbf{y} \| +  \| \mathbf{z} \| + \| \mathbf{z} \| \ge \|  \mathbf{x} \|  + \| \mathbf{y} \|\implies \| \mathbf{x} \|  + \| \mathbf{z} \| +  \| \mathbf{z} \| + \| \mathbf{y} \| \ge \| \mathbf{x} \|  + \| \mathbf{y} \| \implies d(\mathbf{x},\mathbf{y}) \le d(\mathbf{x},\mathbf{z}) + d(\mathbf{z},\mathbf{x})$.\\
\indent First we show that there is a set that is open with $d(\mathbf{x},\mathbf{y})$ but not with $d_{Euclid}(\mathbf{x},\mathbf{y})$. By a theorem in Rudin the set $N(\mathbf{x}) = \{ \mathbf{y} | d(\mathbf{x},\mathbf{y}) < r \}$ is open. For the neighborhood $N(\mathbf{x})$ if we pick an $r < \| \mathbf{x} \|$ (assuming $\mathbf{x} \neq 0$) then since $N(\mathbf{x}) = \{ \mathbf{y} | \| \mathbf{y} \| < r - \| \mathbf{x} \| \}$ for $\mathbf{x} \neq \mathbf{y}$ then since $\|\mathbf{y}\| \ge 0$ there is no $\mathbf{y} \in N(\mathbf{x})$ if $\mathbf{y} \neq \mathbf{x}$. However since $d(\mathbf{x},\mathbf{x}) = 0 < r$ then $\mathbf{x} \in N(\mathbf{x})$. This set with one element, that is $\mathbf{x}$, is open in $d(\mathbf{x},\mathbf{y})$ but is not in $d_{Euclid}(\mathbf{x},\mathbf{y})$.\\
\indent Now we show that every set that is open with respect to $d_{Euclid}(\mathbf{x},\mathbf{y})$ will be open with respect to $d(\mathbf{x},\mathbf{y})$. Now, from the previous discussion it was shown that for all $\mathbf{x} \neq 0$ we can find a neighborhood of $\mathbf{x}$ that consists only of $\mathbf{x}$. So any open set with not containing $0$ can be composed entirely of the one element open sets with the $d(\mathbf{x},\mathbf{y})$ metric which is the definition of being open since these one element sets are neighborhoods. If the neighborhood is around zero then we get the set $N(0) = \{ \mathbf{y} | \| \mathbf{y} < r \}$ which is the same as neighborhood of radius r in $d_{Euclid}(\mathbf{x},\mathbf{y})$. If an open set with the euclidean metric contains 0 then by definition there exists a neighborhood around it contained in the set. Since the neighborhood is the same in both metrics and the remainder of the points can be one-point sets we have shown that every open set in $d_{Euclid}(\mathbf{x},\mathbf{y})$ can be expanded in terms of neighborhoods with the metric $d(\mathbf{x},\mathbf{y})$.

\vspace{15pt}
\begin{flushleft} 
\textbf{Class 18.100B} - Problem 3\\
\rule{500pt}{1pt}\\
\end{flushleft} 

To determine which are the compact sets in $\emph{X}$ we need to determine which are the open sets that can be used to cover a set in $\emph{X}$.
From the definition of open sets it is clear that we need to focus on the neighborhoods that are possible for the given metric. To this end it is clear that there are two types of neighborhoods. For a neighborhood $N_r(x)$ where $r < 1$ the only point that satisfies the requirement on the radius is $x$ itself since $d(x,y) = 0$ if $x = y$ or $d(x,y) = 1$ otherwise.  \\
\indent However if $r \ge 1$ then every point in $\emph(X)$ satisfies the requirement of the neighborhood.\\
\indent From this discussion then it is shown that either a neighborhood consists of a single point or it contains the entire space.

\vspace{15pt}
\begin{flushleft} 
\textbf{Class 18.100B} - Problem 4\\
\rule{500pt}{1pt}\\
\end{flushleft} 

\vspace{15pt}
\begin{flushleft} 
\textbf{Class 18.100B} - Problem 5\\
\rule{500pt}{1pt}\\
\end{flushleft} 

\vspace{15pt}
\begin{flushleft} 
\textbf{Class 18.100B} - Problem 6\\
\rule{500pt}{1pt}\\
\end{flushleft} 

\vspace{15pt}
\begin{flushleft} 
\textbf{Class 18.100B} - Problem 7\\
\rule{500pt}{1pt}\\
\end{flushleft} 

\vspace{15pt}
\begin{flushleft} 
\textbf{Class 18.100B} - Extra Problem \\
\rule{500pt}{1pt}\\
\end{flushleft} 

\end{document}  